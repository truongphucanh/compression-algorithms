Notice that the implementation for compression algorithms was written as readable as possible so the performance may not good.
\paragraph*{Testing process}
For each algorithm we run the encoder, saved the encoded files and calculate the compression ratio as a table. Then, we run the decoder on encoded files to check whether it right or wrong (the same as the original file or not). Some special parameters values would be shown in tables below. RLC and Shannon-Fano Coding were perform on text data and JPEG-Lossless on images data.
\begin{table}[h!]
\begin{center}
\pgfplotstabletypeset[
	multicolumn names, col sep=comma, string type,
	display columns/0/.style={column type={c}},
	display columns/1/.style={column type={c}},
	display columns/2/.style={column type={c}},
	display columns/3/.style={column type={S}},
	every head row/.style={before row={\toprule}, after row={\midrule}},
	every last row/.style={after row=\bottomrule},
]{tables/rlc_ratio.csv} % filename/path to file
\caption{Run-length coding result on text data}
\label{RLC}
\end{center}
\end{table}
\paragraph*{}
As a simplest compression algorithms, RLC works extremely bad on real life document (news, article, mail, etc).Table \ref{RLC} shows that RLC even expends data instead of reduce it(!). The only time RLC really works is the text file that we manually create just for seeing RLC works.
\begin{table}[h!]
\begin{center}
\pgfplotstabletypeset[
	multicolumn names, col sep=comma, string type,
	display columns/0/.style={column type={c}},
	display columns/1/.style={column type={c}},
	display columns/2/.style={column type={c}},
	display columns/3/.style={column type={S}},
	every head row/.style={before row={\toprule}, after row={\midrule}},
	every last row/.style={after row=\bottomrule},
]{tables/shannon_ratio.csv} % filename/path to file
\caption{Shannon Fano coding result on text data}
\label{Shannon}
\end{center}
\end{table}
\paragraph*{}
The Shannon-fano coding is much more better RLC. At least, it really reduce the size of data. The table \ref{Shannon} shows that compression ratio is between 1.3 and 2.4, it seems Shannon-fano works but it's not good enough for what we expected.
\paragraph*{Web Application}
